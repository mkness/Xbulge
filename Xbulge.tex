\documentclass[12pt, preprint]{aastex}

% words
\newcommand{\project}[1]{\textsl{#1}}
\newcommand{\documentname}{\textsl{Article}}


\begin{document}

\title{The underlying X-shaped profile of the bulge revealed by wise} 
\author{D.~Lang\altaffilmark{X},
M.~Ness\altaffilmark{Y},
X.-Y.~Others\altaffilmark{Z},
\textbf{others}}
\altaffiltext{2}{Max-Planck-Institut f\"ur Astronomie, K\"onigstuhl 17, D-69117 Heidelberg, Germany}
\email{dstndstn@gmail.com}

\begin{abstract}%
% Context
% Aims
% Method
% Results
\end{abstract}

\keywords{%
}

\section{Introduction}\label{sec:Intro}

To Do: Science: \\
\ldots: - higher contrast and zoom into inner region.DL 
\ldots: Investigate - symmetry around the major axis = subtraction of resdiuals to do this. According to Victor Debattista there is an interesting contstraint in the symmetry about the midplane - likely as models are asymmetric until they ssettle I'm guessing. 
\ldots: make predictions - mark predictions - as to distribution of stars in X e.g. as a function of age. marie martig will provide this from the simulation and we can use the map to indicate where to look to search for e.g. younger population of stars distributed to the corners of the X. 
\ldots: can also get the angle of the bulge by the ratio of the side of the near and far arms assuming they are the same true size

\section{Method}
- reference existing unwise paper 

\section{Results}  
- what new things do we learn 
- clearest image of the X. 

\section{Experiments and Results}

\ldots 

\acknowledgments
It is a pleasure to thank
XYZ
for valuable discussions and contributions.
This project made use of
  The NASA Astrophysics Data System,
  and open-source code in the \project{numpy} and \project{scipy} packages.


\end{document}
