\documentclass[12pt, preprint]{aastex}

% words
\newcommand{\project}[1]{\textsl{#1}}
\newcommand{\documentname}{\textsl{Article}}


\begin{document}

\title{The Milky Way has an X-shaped bulge} 
\author{D.~Lang\altaffilmark{X},
M.~Ness\altaffilmark{Y},
X.-Y.~Others\altaffilmark{Z},
\textbf{others}}
\altaffiltext{2}{Max-Planck-Institut f\"ur Astronomie, K\"onigstuhl 17, D-69117 Heidelberg, Germany}
\email{dstndstn@gmail.com}

\begin{abstract}%
The Milky Way bulge has a boxy/peanut morphology and an X-shape structure. This X-shape has been revealed by the 
star counts along the line of sight toward the bulge from photometric surveys and a metallicity dependence of this 
`split in the red clump' has been measured by spectroscopic surveys. The X-shaped nature of the bulge has been recently questioned; it has instead been proposed that the apparent split in the red clump is a consequence of stellar populations in an old classical spheroid. We present the WISE imagine of the Milky Way bulge, obtained by coadds of the WISE images, that produce a resolution that is higher than the original WISE public data release. The WISE image of the Milky Way bulge shows that the X-shaped nature of the Milky Way bulge is self-evident and irrefutable. This data used to create this image that we present is publicly available and can be readily accessed by the community for further analysis of the nature of the X-shaped bulge of the Milky Way. 
\end{abstract}


\keywords{%
}

\section{Introduction}\label{sec:Intro}

Using the 3D density of Red Clump Giants (RCGs) mapped by \citet{Wegg2013}  using public
data from the VVV survey the boxy/peanut shape of the bulge was mapped. \citet{Portail2015} use this 3D density map to show that the Milky Way's boxy/peanut bulge has an off-centred X structure, and that the stellar mass involved in the peanut shape accounts for at least 20\% of the stellar mass of the bulge, significantly larger than previously thought. This X-shaped morphology is typical of spiral galaxies that are barred and has been observed in the unsharp masked images of other galaxies \citetp[e.g.][]{Bureau2006} and is a consequence of disk instabilities mapping stars into x-shaped supporting orbits \citep[e.g.][]{Debattista, Inma, Lia}. 

This X-shape was first found by \citet{McWillian} and \citet{Nataf} and later found to be linked to the more metal-rich stars \citep{Nes2012, Uttenthaler2012, RJ2014}. Recently, Lee has called there results into questions and proposed the bulge of the MW is a classical bule and the X=shape is a consequence of helium enrichment of the stars and not a true morphological feature. \citet{Gonzalez2015} has recently provided a detailed analysis of the link between the split in the red clump stars and the X-shaped morphology of the bulge.  The red clump stars are split in their density distribution as a consequence of the spatial density distribution of stars \citep{McWillian, Nataf, Ness, Saito}. We present, for the first time, the WISE image of the Milky Way from \citet{Lang2014a} which clearly demonstrates the Milky Way bulge simply \textit{is} irrefutable, morphologically, X-shaped. This is self evidence and also already unsurprising given all the evidence as summarised in \citet{Gonzalez2015}, from expectations of both dynamical models and also from observations of other galaxies. 


\section{The WISE image}

\begin{figure}[h!]
\centering
        \includegraphics[scale=0.45]{dustinlang_bulge.png}
\caption{WISE image and put link to webpage where this is: http://imagine.legacysurvey.org}
\label{fig:xbulge}
\end{figure}

\section{The WISE image:contrast enhanced}

\begin{figure}[h!]
\centering
        \includegraphics[scale=0.45]{2.png}
\caption{Same as Figure 1 but zoomed in and subtraction of median in each row }
\label{fig:filt}
\end{figure}



References to literature and summary of clump \\
Mention of OG paper reinforcing link \\

\section{Alternative Scenario} 

Summarise the alternative scenario \\

\section{WISE image} 

Put in the main wise image and explain how obtained - reference DL unwise me paper \\

section{enhanced contrast image} 

Put in the enhanced contrast image that Dustin sent 

COBE (Dwek et al. 1995;

because these lines-of-sight pass through both arms of an X-shaped
 E-mail: wegg@mpe.mpg.de
structure which is characteristic of boxy/peanut (B/P) bulges
in barred galaxies (McWilliam & Zoccali 2010; Ness et al.
2012).

From Gonzalez 2015: Recently, Lee et al. (2015) presented a different interpretation
for the split RC in which a classical bulge with an additional
population enriched in helium co-exists with a bar. Within this
hypothesis, the authors assigned all the bar-like properties to the
Milky Way bar component, which has not undergone a buckling
instability and is thus restricted to low Galactic latitudes.
The double RC is then not caused by the X-shape of the bar, but
instead is the consequence of a massive classical bulge with a
significant fraction of stars enriched in helium. In this letter, we
challenge the validity of this scenario based on the observational
properties of the RC in some specific lines of sight.




Portail 2015b We analyze N-body models of barred discs evolved from a near
equilibrium stellar disc embedded in different live dark matter halos.
During this evolution the disc naturally forms a bar which
rapidly buckles out of the Galactic plane and creates a B/P bulge
(Combes & Sanders 1981; Raha et al. 1991)

"While it is incontrovertible that the inner Galaxy contains a bar," Wegg 2013

Seen in simulations e.g. making it vertically thick and creating the so
called Box/Peanut shape, or X-shape in unsharp-masked images.
o (Debattista & Sellwood 2000; Athanassoula
2003). Inma2006

Xshape seen in external galaxies with unsharp masking e.g. Bureau 2006

\section {Conclusions} 

Point out again self evident from this data and data is available. 

Put in two papers - one just showing the x and one showing the high contrast. 

%To Do: Science: \\
%\ldots: - higher contrast and zoom into inner region.DL 
%\ldots: Investigate - symmetry around the major axis = subtraction of resdiuals to do this. According to Victor Debattista there is an interesting contstraint in the symmetry about the midplane - likely as models are asymmetric until they ssettle I'm guessing. 
%\ldots: make predictions - mark predictions - as to distribution of stars in X e.g. as a function of age. marie martig will provide this from the simulation and we can use the map to indicate where to look to search for e.g. younger population of stars distributed to the corners of the X. 
%\ldots: can also get the angle of the bulge by the ratio of the side of the near and far arms assuming they are the same true size

\section{Method}
- reference existing unwise paper 

\section{Results}  
- what new things do we learn 
- clearest image of the X. 


\section{Experiments and Results}

\ldots 

\acknowledgments
It is a pleasure to thank
XYZ
for valuable discussions and contributions.
This project made use of
  The NASA Astrophysics Data System,
  and open-source code in the \project{numpy} and \project{scipy} packages.


\end{document}
