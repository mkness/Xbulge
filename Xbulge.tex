\documentclass[12pt, preprint]{aastex}

% words
\newcommand{\project}[1]{\textsl{#1}}
\newcommand{\documentname}{\textsl{Article}}


\begin{document}

\title{The Milky Way has an X-shaped bulge} 
\author{D.~Lang\altaffilmark{X},
M.~Ness\altaffilmark{Y},
X.-Y.~Others\altaffilmark{Z},
\textbf{others}}
\altaffiltext{2}{Max-Planck-Institut f\"ur Astronomie, K\"onigstuhl 17, D-69117 Heidelberg, Germany}
\email{dstndstn@gmail.com}

\begin{abstract}%
The Milky Way bulge has a boxy/peanut morphology and an X-shape structure. This X-shape has been revealed by the 
star counts along the line of sight toward the bulge from photometric surveys and a metallicity dependence of this 
`split in the red clump' has been measured by spectroscopic surveys. The X-shaped nature of the bulge has been recently questioned and it has instead been proposed that the apparent split in the red clump is a consequence of stellar populations in an old classical spheroid. We present the WISE imagine of the Milky Way bulge, obtained from the <data> with <an improved statistical treatment> that <has enabled higher resolution>. This image shows that the X-shaped nature of this bulge is self-evident and irrefutable. This data we present is publicly available and can be readily accessed by the community for analysis of the nature of the X-shaped bulge of the Milky Way. 
\end{abstract}


\keywords{%
}

\section{Introduction}\label{sec:Intro}

Put in two papers - one just showing the x and one showing the high contrast. 

%To Do: Science: \\
%\ldots: - higher contrast and zoom into inner region.DL 
%\ldots: Investigate - symmetry around the major axis = subtraction of resdiuals to do this. According to Victor Debattista there is an interesting contstraint in the symmetry about the midplane - likely as models are asymmetric until they ssettle I'm guessing. 
%\ldots: make predictions - mark predictions - as to distribution of stars in X e.g. as a function of age. marie martig will provide this from the simulation and we can use the map to indicate where to look to search for e.g. younger population of stars distributed to the corners of the X. 
%\ldots: can also get the angle of the bulge by the ratio of the side of the near and far arms assuming they are the same true size

\section{Method}
- reference existing unwise paper 

\section{Results}  
- what new things do we learn 
- clearest image of the X. 

\section{Experiments and Results}

\ldots 

\acknowledgments
It is a pleasure to thank
XYZ
for valuable discussions and contributions.
This project made use of
  The NASA Astrophysics Data System,
  and open-source code in the \project{numpy} and \project{scipy} packages.


\end{document}
